%%%%%%%%%%%%%%%%%%%%%%%%%%%%%%%%%%%%%%%%%%%%%%%%%%
% This template was created by Sean Peneyra.  Share freely.
%%%%%%%%%%%%%%%%%%%%%%%%%%%%%%%%%%%%%%%%%%%%%%%%%%

%      Some common commands using the packages below. 
%      Copy and paste the code below the description. Then remove the percentage signs:

%     Description: Add an image (make sure your image file is in the same folder as this tex file
%\begin{figure}[htbp]
%\scalebox{.4}{\includegraphics{image.jpg}}
%\caption{Add a caption here}
%\end{figure}

%     Description: Add a code file.  Useful for correctly formatting external files
%\lstinputlisting[language=Matlab]{code.m}

\documentclass[letterpaper,notitlepage]{article}
\usepackage[top=1.4in, bottom=1.4in, left=1.35in, right=1.35in]{geometry}
\usepackage[english]{babel}
\usepackage[utf8]{inputenc}
\usepackage{amssymb}
\usepackage{amsmath}
\usepackage{array}
\usepackage{listings}
\usepackage{xcolor}
\usepackage{graphicx}
\usepackage{fancyhdr}
\usepackage{mathrsfs}
\usepackage{tikz}
\definecolor{mygreen}{rgb}{0,0.6,0}
\definecolor{mygray}{rgb}{0.5,0.5,0.5}
\definecolor{mymauve}{rgb}{0.58,0,0.82}
\lstset{language=C,backgroundcolor=\color{white},basicstyle=\footnotesize,breakatwhitespace=false,breaklines=true,captionpos=b,commentstyle=\color{mygreen}, keywordstyle=\color{blue},numbers=left,numbersep=5pt,numberstyle=\tiny\color{mygray},rulecolor=\color{black},stringstyle=\color{mymauve},tabsize=2,title=\lstname}
\pagestyle{fancy}
\fancyhf{}

%%%%%%%%%%%%%%%%%%%%%%%%%%%%%%%%%%%%%%%%%%%%%%%%%%
% Change rhead to name, Change lhead to Homework # or Title
\rhead{Sean Peneyra}
\lhead{Homework 2}

\rfoot{\thepage}
\begin{document}

%%%%%%%%%%%%%%%%%%%%%%%%%%%%%%%%%%%%%%%%%%%%%%%%%%
% Change title to Class name. \\ indicates a line break.  Use the \Large{} to add subtitle
\title{MATH 645 - Survey of Math Problems I\\ \Large{Homework 2}}

%%%%%%%%%%%%%%%%%%%%%%%%%%%%%%%%%%%%%%%%%%%%%%%%%%
% Change author and date accordingly
\author{Sean Peneyra}
\date{September 11, 2019}

\maketitle
\newcommand{\multideg}[1]{\text{multideg}(#1)}
\newcommand{\LT}[1]{\small{\textsc{LT}(#1)}}
\newcommand{\LM}[1]{\small{\textsc{LM}(#1)}}
\newcommand{\LC}[1]{\small{\textsc{LC}(#1)}}
\newcommand{\posint}[0]{\mathbb Z^n_{\geq 0}}
\newcommand{\lex}[0]{>_{lex}}
\newcommand{\grlex}[0]{>_{grlex}}
\newcommand{\grevlex}[0]{>_{grevlex}}
\newcommand{\braket}[4]{\langle #1\:#2|#3\:#4\rangle}
\newcommand{\Shat}[1]{\hat{S}_{#1}}
\newcommand{\twovec}[2]{\left[\!\begin{array}{c}#1\\#2\end{array}\!\right]}
\newcommand{\twoform}[2]{\left[\begin{array}{cc}#1&#2\end{array}\right]}
\newcommand{\threevec}[3]{\left[\begin{array}{c}#1\\#2\\#3\end{array}\right]}
\newcommand{\threeform}[3]{\left[\begin{array}{ccc}#1&#2&#3\end{array}\right]}
\newcommand{\twomatrix}[4]{\left[\begin{array}{cc}#1&#2\\#3&#4\end{array}\right]}
\newcommand{\oc}[0]{\twomatrix{1}{0}{0}{1}}
\newcommand{\ic}[0]{\twomatrix{i}{0}{0}{-i}}
\newcommand{\jc}[0]{\twomatrix{0}{1}{-1}{0}}
\newcommand{\kc}[0]{\twomatrix{0}{i}{i}{0}}
\newcommand{\ob}[0]{\mathbf{1}}
\newcommand{\ib}[0]{\mathbf{i}}
\newcommand{\jb}[0]{\mathbf{j}}
\newcommand{\kb}[0]{\mathbf{k}}
\newcommand{\threematrix}[9]{\left[\begin{array}{ccc}#1&#2&#3\\#4&#5&#6\\#7&#8&#9\end{array}\right]}
\newcommand{\ma}[1]{\measuredangle #1}
\newcommand{\Amatri}[0]{\threematrix{e^t}{2e^{-t}}{e^{2t}}{-e^t}{2e^{-t}}{e^{2t}}{3e^t}{-e^{-t}}{-e^{2t}}}
\newcommand{\Bmatri}[0]{\threematrix{2e^t}{e^{-t}}{3e^{2t}}{2e^t}{e^{-t}}{-e^{2t}}{-e^t}{3e^{-t}}{2e^{2t}}}
\newcommand{\del}[0]{\partial}
\newcommand{\delx}[1]{\frac{\partial #1}{\partial x}}
\newcommand{\dely}[1]{\frac{\partial #1}{\partial y}}
\newcommand{\bey}[0]{\begin{equation}}
\newcommand{\eey}[0]{\end{equation}}
\newcommand{\ben}[0]{\begin{equation}}
\newcommand{\een}[0]{\end{equation}}
\newcommand{\bay}[0]{\begin{align}}
\newcommand{\eay}[0]{\end{align}}
\newcommand{\ban}[0]{\begin{align}}
\newcommand{\ean}[0]{\end{align}}
\setcounter{section}{0}

%%%%%%%%%%%%%%%%%%%%%%%%%%%%%%%%%%%%%%%%%%%%%%%%%%
% Your paper starts here. Put a blurb before you start the first \section if you like
\section{Part I.}


%%%%%%%%%%%%%%%%%%%%%%%%%%%%%%%%%%%%%%%%%%%%%%%%%%
% bibliography starts here.  update the number of bibitems after (thebibliography)
\newpage
\begin{thebibliography}{1}
\bibitem{Book1}
Book information
\end{thebibliography}
\end{document}
